\documentclass[11pt]{article}

%%% These are some packages that are useful
\usepackage{circuitikz} %used for circuit diagrams
\usepackage{lastpage} %allows us to determine how many total pages there are
\usepackage{amsfonts, lipsum}
\usepackage{amsmath,amssymb, amscd,amsbsy, amsthm, enumerate}
\usepackage{mdframed, titlesec, setspace,verbatim, multicol}
\usepackage[top=1in, bottom=1in, left=1in, right=1in]{geometry} %sets the margins
\usepackage[unicode]{hyperref} %enhanced references
\usepackage{tikz, pgfplots, xcolor}
\usepackage{fancyhdr} %creates a fancy header for labeling document/name/date 
\usepackage{listings}
\usepackage{xcolor}
\usepackage{vwcol}  
\usepackage{enumitem}

%%% Page formatting
%\setlength{\headsep}{30pt}
\setlength{\parindent}{25pt}
\setlength{\textheight}{9in}

%%% Header and Footer Info
\pagestyle{fancy}
\fancyhead[L]{{\footnotesize \large PHY 482 - \textbf{Homework 09}}}
\fancyhead[C]{\footnotesize \today}
\fancyhead[R]{\footnotesize Names: Antonius Torode \& Eric Aboud}
\fancyfoot[L]{}
\fancyfoot[C]{}
\fancyfoot[R]{\thepage\ of \pageref{LastPage}}

%%Use these commands to clear the fancyhdr settings and horizontal header line.
%\fancyhf{} % sets both header and footer to nothing
%\renewcommand{\headrulewidth}{0pt}


%%% This defines the solution environment for you to write your solutions
\newenvironment{soln}
{\let\oldqedsymbol=\qedsymbol
	\renewcommand{\qedsymbol}{$ $}
	\begin{proof}[\bfseries\upshape \color{blue}Solution]\color{blue}}
	{\end{proof}
	\renewcommand{\qedsymbol}{\oldqedsymbol}}


%%% Document Starts now
\begin{document} 

\begin{center}
	{\Large E\&M II - Project 2 outline}
\end{center}

For the project we plan to calculate the hyperfine structure (Zeeman effect) for Hydrogen and
another atom. Since these calculations become exponentially more involved with the difficulty of
the atomic nuclei structure, we will perhaps try to look at simple atoms such as deuterium. We
will then find how different magnetic fields affect the structure and determine if any anomalies
occur for different magnetic fields. It will be up to both of us together to complete these
calculation (since they are involved).

The largest part of the project will be creating methods of calculating the Hamiltoniann. This will
probably take the most time, which we could hopefully expect to have completed in a couple
weeks (since <Eric> may be gone multiple times to visit grad schools). Once that is complete
we should take another week to compute the effect that different magnetic fields have on the
structure. We will numerically or graphically represent the results for a range of cases with
different magnetic field strengths. We will start with the simple hydrogen atom calculations and
determine the correction terms due to a magnetic field. From this, we can determine the
Hamiltonian and use perturbation theory to solve for energy and wave function corrections. This
may prove different for higher energy states of Hydrogen due to degeneracies which is where a
computer system will be applied to help solve for the corrections. We may only be able to
achieve results for a couple low level n states but we will see.

If we have time at the end, we can look at how a linearly changing magnetic field can induce an
electric field. Since a time dependent magnetic field will cause an induced changing electric
field, this will cause the stark effect to play a role in the corrections to the Hamiltonian. This
would be very complicated on it’s own so we can perhaps look at a specific case for the
hydrogen atom, such as the ground state and choose a simple magnetic field changing in time
to explore how this affects the perturbations. This is where we can involve much more E\&M into
our project, but I suspect it will be far too much work to try and generalize the effects of a
changing magnetic field any.


%%%%%%%%%%%%%%%%%%%%%%%%%%%%%%%%%%%%%%%%%%%%%%%%%%%%%%%%%%%%%%%%%%%%%%%%%%%%%%%%%%%%%%%%%%%

%%%%%%%%%%%%%%%%%%%%%%%%%%%%%%%%%%%%%%%%%%%%%%%%%%%%%%%%%%%%%%%%%%%%%%%%%%%%%%%%%%%%%%%%%%%
\end{document}





















