\documentclass[11pt]{article}

%%% These are some packages that are useful
\usepackage{circuitikz} %used for circuit diagrams
\usepackage{lastpage} %allows us to determine how many total pages there are
\usepackage{amsfonts, lipsum}
\usepackage{amsmath,amssymb, amscd,amsbsy, amsthm, enumerate}
\usepackage{mdframed, titlesec, setspace,verbatim, multicol}
\usepackage[top=1in, bottom=1in, left=1in, right=1in]{geometry} %sets the margins
\usepackage[unicode]{hyperref} %enhanced references
\usepackage{tikz, pgfplots, xcolor}
\usepackage{fancyhdr} %creates a fancy header for labeling document/name/date 
\usepackage{listings}
\usepackage{xcolor}
\usepackage{vwcol}  
\usepackage{enumitem}

%%% Page formatting
%\setlength{\headsep}{30pt}
\setlength{\parindent}{25pt}
\setlength{\textheight}{9in}

%%% Header and Footer Info
\pagestyle{fancy}
\fancyhead[L]{{\footnotesize \large PHY 482 - calculations}}
\fancyhead[C]{\footnotesize \today}
\fancyhead[R]{\footnotesize Name: Antonius Torode \& Eric Aboud}
\fancyfoot[L]{}
\fancyfoot[C]{}
\fancyfoot[R]{\thepage\ of \pageref{LastPage}}

%%Use these commands to clear the fancyhdr settings and horizontal header line.
%\fancyhf{} % sets both header and footer to nothing
%\renewcommand{\headrulewidth}{0pt}


%%% This defines the solution environment for you to write your solutions
\newenvironment{soln}
{\let\oldqedsymbol=\qedsymbol
	\renewcommand{\qedsymbol}{$ $}
	\begin{proof}[\bfseries\upshape \color{blue}Solution]\color{blue}}
	{\end{proof}
	\renewcommand{\qedsymbol}{\oldqedsymbol}}


%%% Document Starts now
\begin{document} 

\begin{center}
	{\Large E\&M II - Project 2 Calculations.}
\end{center}

The \textbf{Zeeman effect} is an effect that comes about from putting an atom in a magnetic field. The magnetic field causes a magnetic moment to be produced and perturbs the energy levels of the atom causing spectral line shifts to appear when modeled or measured with spectroscopy. 

\section{Hydrogen Atom First Order Magnetic Field Energy Corrections}
To begin, the unperturbed wave functions of hydrogen is given by 
\begin{align}
H^{(0)} = \underbrace{\frac{\vec{p}^2}{2m}}_{\textrm{Kinetic term}}-\underbrace{\frac{\hbar c \alpha}{r}}_{\textrm{Potential term}}.
\end{align}
Consider the atom now in a magnetic field $\vec{B}$. The electrons spin will yield a magnetic moment $\vec{\mu}_S=\frac{-eg_s}{2m}\vec{S}$. The energy interaction of a single magnetic moment $\vec{\mu}$ with a fixed external magnetic field is given by $H^{\mu}= - \mu \cdot \vec{B}$ and so the Hamiltonian change due to this effect is $H_{S}^{(1)} =  -\mu_s \cdot \vec{B} = \frac{eg_s}{2m}\vec{S}\cdot \vec{B}$. There is also a correction due to the angular momentum of the electron. If we assume that the electron is moving in a circle, then this has a magnetic dipole $\mu_L = \frac{-e}{2m}\vec{L}$ and so there exists another correction to the Hamiltonian $H_L^{(1)} = -\mu_L \cdot\vec{B} = \frac{eg_L}{2m}\vec{L}\cdot \vec{B}$. Now, $g_s$ is a bi-products of the Dirac equation has a value of $g_s=s$. Combining these terms together give the total Zeeman effect to the Hamiltonian $H_{magnetic}^{(1)}$,
\begin{align}
H_{magnetic}^{(1)} \equiv H_{mag}^{(1)}= \frac{e}{2m}(\vec{L}+2\vec{S})\cdot \vec{B} = \frac{\mu_B}{\hbar}(\vec{L}+2\vec{S})\cdot \vec{B},
\end{align} 
where $\mu_B=\frac{e\hbar}{2m}$ is the Bohr magneton.

If we are just considering the changes that come about from the Zeeman effect, we can ignore other Hamiltonian perturbations such as the fine structure and hyperfine structure interactions. Assume that $\vec{B} \equiv B\hat{z}$ then the Hamiltonian will only contain z components after performing the dot product giving $H_{mag}^{(1)} = \frac{\mu_B}{\hbar}(L_z+2S_z)B$. Now, $\vec{J}=\vec{L}+\vec{S} \implies L_z = J_z-S_z$ and so
$H_{mag}^{(1)} = \frac{\mu_B}{\hbar}(J_z+S_z)B$. From perturbation theory we can solve for the first order energy corrections to this using
\begin{align}
E_{n}^{(1)} &= \langle n j m_j \ell s| H_{mag}^{(1)}|n j m_j \ell s \rangle \nonumber\\
&= \frac{\mu_BB}{\hbar}\langle n j m_j \ell s| (J_z+S_z)|n j m_j \ell s \rangle \nonumber\\
&= \frac{\mu_BB}{\hbar}\left(\hbar m_j+\langle n j m_j \ell s| S_z|n j m_j \ell s \rangle \right). \label{<njmls|S_z|nlmls>}
\end{align}
We can determine $S_z$ by projecting $\vec{S}$ onto $\vec{J}$ which gives $\vec{S} = \frac{\vec{J}\cdot\vec{S}}{\vec{J}^2}\vec{J}$ and then using the substitution that 
\begin{align}
\vec{J}\cdot \vec{S} = (\vec{L}+\vec{S})\cdot \vec{S} = \vec{L}\cdot\vec{S}+\vec{S}^2 = \frac{1}{2}(\vec{J^2}-\vec{L^2}-\vec{S}^2)+\vec{S^2} = \frac{1}{2}(\vec{J^2}-\vec{L^2}+\vec{S}^2).
\end{align}
This, we have
\begin{align}
S_z = \frac{(\vec{J^2}-\vec{L^2}+\vec{S}^2)}{2\vec{J}^2}J_z,
\end{align}
and so the undetermined matrix element in (\ref{<njmls|S_z|nlmls>}) becomes 
\begin{align}
\langle n j m_j \ell s| S_z|n j m_j \ell s \rangle &= \langle n j m_j \ell s| \frac{(\vec{J^2}-\vec{L^2}+\vec{S}^2)}{2\vec{J}^2}J_z|n j m_j \ell s \rangle \nonumber \\
&= m_j \hbar \frac{j(j+1)-\ell(\ell+1)+s(s+1)}{2j(j+1)}.
\end{align}
Plugging this back into (\ref{<njmls|S_z|nlmls>}) gives us 
\begin{align}
\boxed{E_{n}^{(1)} = \mu_BBm_j\left[1+\frac{j(j+1)-\ell(\ell+1)+s(s+1)}{2j(j+1)}\right] = \mu_BBm_jg_L}, \label{E_{n}^{(1)}}
\end{align}
where $g_L$ is the Land\'{e} g factor
\begin{align}
g_L = 1+\frac{j(j+1)-\ell(\ell+1)+s(s+1)}{2j(j+1)}.
\end{align}

\begin{thebibliography}{9}
	\bibitem{brain} Our brains.
\end{thebibliography}
%%%%%%%%%%%%%%%%%%%%%%%%%%%%%%%%%%%%%%%%%%%%%%%%%%%%%%%%%%%%%%%%%%%%%%%%%%%%%%%%%%%%%%%%%%%

%%%%%%%%%%%%%%%%%%%%%%%%%%%%%%%%%%%%%%%%%%%%%%%%%%%%%%%%%%%%%%%%%%%%%%%%%%%%%%%%%%%%%%%%%%%
\end{document}





















